%% Generated by Sphinx.
\def\sphinxdocclass{report}
\documentclass[letterpaper,10pt,english]{sphinxmanual}
\ifdefined\pdfpxdimen
   \let\sphinxpxdimen\pdfpxdimen\else\newdimen\sphinxpxdimen
\fi \sphinxpxdimen=.75bp\relax

\PassOptionsToPackage{warn}{textcomp}
\usepackage[utf8]{inputenc}
\ifdefined\DeclareUnicodeCharacter
% support both utf8 and utf8x syntaxes
  \ifdefined\DeclareUnicodeCharacterAsOptional
    \def\sphinxDUC#1{\DeclareUnicodeCharacter{"#1}}
  \else
    \let\sphinxDUC\DeclareUnicodeCharacter
  \fi
  \sphinxDUC{00A0}{\nobreakspace}
  \sphinxDUC{2500}{\sphinxunichar{2500}}
  \sphinxDUC{2502}{\sphinxunichar{2502}}
  \sphinxDUC{2514}{\sphinxunichar{2514}}
  \sphinxDUC{251C}{\sphinxunichar{251C}}
  \sphinxDUC{2572}{\textbackslash}
\fi
\usepackage{cmap}
\usepackage[T1]{fontenc}
\usepackage{amsmath,amssymb,amstext}
\usepackage{babel}



\usepackage{times}
\expandafter\ifx\csname T@LGR\endcsname\relax
\else
% LGR was declared as font encoding
  \substitutefont{LGR}{\rmdefault}{cmr}
  \substitutefont{LGR}{\sfdefault}{cmss}
  \substitutefont{LGR}{\ttdefault}{cmtt}
\fi
\expandafter\ifx\csname T@X2\endcsname\relax
  \expandafter\ifx\csname T@T2A\endcsname\relax
  \else
  % T2A was declared as font encoding
    \substitutefont{T2A}{\rmdefault}{cmr}
    \substitutefont{T2A}{\sfdefault}{cmss}
    \substitutefont{T2A}{\ttdefault}{cmtt}
  \fi
\else
% X2 was declared as font encoding
  \substitutefont{X2}{\rmdefault}{cmr}
  \substitutefont{X2}{\sfdefault}{cmss}
  \substitutefont{X2}{\ttdefault}{cmtt}
\fi


\usepackage[Bjarne]{fncychap}
\usepackage{sphinx}

\fvset{fontsize=\small}
\usepackage{geometry}

% Include hyperref last.
\usepackage{hyperref}
% Fix anchor placement for figures with captions.
\usepackage{hypcap}% it must be loaded after hyperref.
% Set up styles of URL: it should be placed after hyperref.
\urlstyle{same}
\addto\captionsenglish{\renewcommand{\contentsname}{Contents:}}

\usepackage{sphinxmessages}
\setcounter{tocdepth}{1}



\title{Keras Video Generator}
\date{Nov 22, 2019}
\release{}
\author{Patrice Ferlet}
\newcommand{\sphinxlogo}{\vbox{}}
\renewcommand{\releasename}{}
\makeindex
\begin{document}

\pagestyle{empty}
\sphinxmaketitle
\pagestyle{plain}
\sphinxtableofcontents
\pagestyle{normal}
\phantomsection\label{\detokenize{index::doc}}

\index{keras\_video (module)@\spxentry{keras\_video}\spxextra{module}}

\chapter{Video Generator package}
\label{\detokenize{index:video-generator-package}}
Provides generators for video sequence that can be injected in Time Distributed layer.
\phantomsection\label{\detokenize{index:module-keras_video.generator}}\index{keras\_video.generator (module)@\spxentry{keras\_video.generator}\spxextra{module}}

\chapter{VideoFrameGenerator - Simple Generator}
\label{\detokenize{index:videoframegenerator-simple-generator}}
A simple frame generator that takes distributed frames from
videos. It is useful for videos that are scaled from frame 0 to end
and that have no noise frames.
\index{VideoFrameGenerator (class in keras\_video.generator)@\spxentry{VideoFrameGenerator}\spxextra{class in keras\_video.generator}}

\begin{fulllineitems}
\phantomsection\label{\detokenize{index:keras_video.generator.VideoFrameGenerator}}\pysiglinewithargsret{\sphinxbfcode{\sphinxupquote{class }}\sphinxcode{\sphinxupquote{keras\_video.generator.}}\sphinxbfcode{\sphinxupquote{VideoFrameGenerator}}}{\emph{rescale=0.00392156862745098}, \emph{nb\_frames: int = 5}, \emph{classes: list = {[}{]}}, \emph{batch\_size: int = 16}, \emph{use\_frame\_cache: bool = False}, \emph{target\_shape: tuple = (224}, \emph{224)}, \emph{shuffle: bool = True}, \emph{transformation: keras.preprocessing.image.ImageDataGenerator = None}, \emph{split: float = None}, \emph{nb\_channel: int = 3}, \emph{glob\_pattern: str = './videos/\{classname\}/*.avi'}, \emph{\_validation\_data: list = None}}{}
Create a generator that return batches of frames from video
\begin{itemize}
\item {} 
rescale: float fraction to rescale pixel data (commonly 1/255.)

\item {} 
nb\_frames: int, number of frames to return for each sequence

\item {} 
classes: list of str, classes to infer

\item {} 
batch\_size: int, batch size for each loop

\item {} \begin{description}
\item[{use\_frame\_cache: bool, use frame cache (may take a lot of memory for}] \leavevmode
large dataset)

\end{description}

\item {} 
shape: tuple, target size of the frames

\item {} 
shuffle: bool, randomize files

\item {} 
transformation: ImageDataGenerator with transformations

\item {} 
split: float, factor to split files and validation

\item {} 
nb\_channel: int, 1 or 3, to get grayscaled or RGB images

\item {} \begin{description}
\item[{glob\_pattern: string, directory path with ‘\{classname\}’ inside that}] \leavevmode
will be replaced by one of the class list

\end{description}

\item {} 
\_validation\_data: already filled list of data, do not touch !

\end{itemize}

You may use the “classes” property to retrieve the class list afterward.

The generator has that properties initialized:
\begin{itemize}
\item {} 
classes\_count: number of classes that the generator manages

\item {} 
files\_count: number of video that the generator can provides

\item {} 
classes: the given class list

\item {} \begin{description}
\item[{files: the full file list that the generator will use, this}] \leavevmode
is usefull if you want to remove some files that should not be
used by the generator.

\end{description}

\end{itemize}
\index{get\_validation\_generator() (keras\_video.generator.VideoFrameGenerator method)@\spxentry{get\_validation\_generator()}\spxextra{keras\_video.generator.VideoFrameGenerator method}}

\begin{fulllineitems}
\phantomsection\label{\detokenize{index:keras_video.generator.VideoFrameGenerator.get_validation_generator}}\pysiglinewithargsret{\sphinxbfcode{\sphinxupquote{get\_validation\_generator}}}{}{}
Return the validation generator if you’ve provided split factor

\end{fulllineitems}

\index{on\_epoch\_end() (keras\_video.generator.VideoFrameGenerator method)@\spxentry{on\_epoch\_end()}\spxextra{keras\_video.generator.VideoFrameGenerator method}}

\begin{fulllineitems}
\phantomsection\label{\detokenize{index:keras_video.generator.VideoFrameGenerator.on_epoch_end}}\pysiglinewithargsret{\sphinxbfcode{\sphinxupquote{on\_epoch\_end}}}{}{}
Called by Keras after each epoch

\end{fulllineitems}


\end{fulllineitems}

\phantomsection\label{\detokenize{index:module-keras_video.sliding}}\index{keras\_video.sliding (module)@\spxentry{keras\_video.sliding}\spxextra{module}}

\chapter{Sliding frames}
\label{\detokenize{index:sliding-frames}}
That module provides the SlidingFrameGenerator that is helpful
to get more sequence from one video file. The goal is to provide decayed
sequences for the same action.
\index{SlidingFrameGenerator (class in keras\_video.sliding)@\spxentry{SlidingFrameGenerator}\spxextra{class in keras\_video.sliding}}

\begin{fulllineitems}
\phantomsection\label{\detokenize{index:keras_video.sliding.SlidingFrameGenerator}}\pysiglinewithargsret{\sphinxbfcode{\sphinxupquote{class }}\sphinxcode{\sphinxupquote{keras\_video.sliding.}}\sphinxbfcode{\sphinxupquote{SlidingFrameGenerator}}}{\emph{*args}, \emph{sequence\_time: int = None}, \emph{**kwargs}}{}
SlidingFrameGenerator is useful to get several sequence of
the same “action” by sliding the cursor of video. For example, with a
video that have 60 frames using 30 frames per second, and if you want
to pick 6 frames, the generator will return:
\begin{itemize}
\item {} 
one sequence with frame \sphinxtitleref{{[} 0,  5, 10, 15, 20, 25{]}}

\item {} 
then \sphinxtitleref{{[} 1,  6, 11, 16, 21, 26{]})}

\item {} 
and so on to frame 30

\end{itemize}

If you set \sphinxtitleref{sequence\_time} parameter, so the sequence will be reduce to
the given time.

params:
\begin{itemize}
\item {} \begin{description}
\item[{sequence\_time: int seconds of the sequence to fetch, if None, the entire}] \leavevmode
vidoe time is used

\end{description}

\end{itemize}

from VideoFrameGenerator:
\begin{itemize}
\item {} 
rescale: float fraction to rescale pixel data (commonly 1/255.)

\item {} 
nb\_frames: int, number of frames to return for each sequence

\item {} 
classes: list of str, classes to infer

\item {} 
batch\_size: int, batch size for each loop

\item {} \begin{description}
\item[{use\_frame\_cache: bool, use frame cache (may take a lot of memory for}] \leavevmode
large dataset)

\end{description}

\item {} 
shape: tuple, target size of the frames

\item {} 
shuffle: bool, randomize files

\item {} 
transformation: ImageDataGenerator with transformations

\item {} 
split: float, factor to split files and validation

\item {} 
nb\_channel: int, 1 or 3, to get grayscaled or RGB images

\item {} \begin{description}
\item[{glob\_pattern: string, directory path with ‘\{classname\}’ inside that}] \leavevmode
will be replaced by one of the class list

\end{description}

\end{itemize}
\index{get\_validation\_generator() (keras\_video.sliding.SlidingFrameGenerator method)@\spxentry{get\_validation\_generator()}\spxextra{keras\_video.sliding.SlidingFrameGenerator method}}

\begin{fulllineitems}
\phantomsection\label{\detokenize{index:keras_video.sliding.SlidingFrameGenerator.get_validation_generator}}\pysiglinewithargsret{\sphinxbfcode{\sphinxupquote{get\_validation\_generator}}}{}{}
Return the validation generator if you’ve provided split factor

\end{fulllineitems}

\index{on\_epoch\_end() (keras\_video.sliding.SlidingFrameGenerator method)@\spxentry{on\_epoch\_end()}\spxextra{keras\_video.sliding.SlidingFrameGenerator method}}

\begin{fulllineitems}
\phantomsection\label{\detokenize{index:keras_video.sliding.SlidingFrameGenerator.on_epoch_end}}\pysiglinewithargsret{\sphinxbfcode{\sphinxupquote{on\_epoch\_end}}}{}{}
Called by Keras after each epoch

\end{fulllineitems}


\end{fulllineitems}

\phantomsection\label{\detokenize{index:module-keras_video.flow}}\index{keras\_video.flow (module)@\spxentry{keras\_video.flow}\spxextra{module}}

\chapter{Optical Flow Generator}
\label{\detokenize{index:optical-flow-generator}}
\begin{sphinxadmonition}{warning}{Warning:}
This module is not stable !
\end{sphinxadmonition}

The purpose of that module is to return optical flow sequences from a video.

Several methods are defined:
\begin{itemize}
\item {} \begin{description}
\item[{Use standard optical flow}] \leavevmode
METHOD\_OPTICAL\_FLOW=1

\end{description}

\item {} \begin{description}
\item[{Use optical flow as a mask on video}] \leavevmode
METHOD\_FLOW\_MASK=2

\end{description}

\item {} \begin{description}
\item[{Use absolute diff mask on video}] \leavevmode
METHOD\_DIFF\_MASK=3

\end{description}

\item {} \begin{description}
\item[{Use abs diff}] \leavevmode
METHOD\_ABS\_DIFF=4

\end{description}

\end{itemize}
\index{OpticalFlowGenerator (class in keras\_video.flow)@\spxentry{OpticalFlowGenerator}\spxextra{class in keras\_video.flow}}

\begin{fulllineitems}
\phantomsection\label{\detokenize{index:keras_video.flow.OpticalFlowGenerator}}\pysiglinewithargsret{\sphinxbfcode{\sphinxupquote{class }}\sphinxcode{\sphinxupquote{keras\_video.flow.}}\sphinxbfcode{\sphinxupquote{OpticalFlowGenerator}}}{\emph{*args}, \emph{nb\_frames=5}, \emph{method=1}, \emph{flowlevel=3}, \emph{iterations=3}, \emph{winsize=15}, \emph{**kwargs}}{}
Generate optical flow sequence from frames in videos. It can
use different methods.

params:
\begin{itemize}
\item {} \begin{description}
\item[{method: METHOD\_OPTICAL\_FLOW, METHOD\_FLOW\_MASK, METHOD\_DIFF\_MASK,}] \leavevmode
METHOD\_ABS\_DIFF

\end{description}

\item {} 
flowlevel: integer that give the flow level to calcOpticalFlowFarneback

\item {} 
iterations: integer number of iterations for calcOpticalFlowFarneback

\item {} 
winsize: flow window size for calcOpticalFlowFarneback

\end{itemize}

from VideoFrameGenerator:
\begin{itemize}
\item {} 
rescale: float fraction to rescale pixel data (commonly 1/255.)

\item {} 
nb\_frames: int, number of frames to return for each sequence

\item {} 
classes: list of str, classes to infer

\item {} 
batch\_size: int, batch size for each loop

\item {} \begin{description}
\item[{use\_frame\_cache: bool, use frame cache (may take a lot of memory for}] \leavevmode
large dataset)

\end{description}

\item {} 
shape: tuple, target size of the frames

\item {} 
shuffle: bool, randomize files

\item {} 
transformation: ImageDataGenerator with transformations

\item {} 
split: float, factor to split files and validation

\item {} 
nb\_channel: int, 1 or 3, to get grayscaled or RGB images

\item {} \begin{description}
\item[{glob\_pattern: string, directory path with ‘\{classname\}’ inside that}] \leavevmode
will be replaced by one of the class list

\end{description}

\end{itemize}
\index{absdiff() (keras\_video.flow.OpticalFlowGenerator method)@\spxentry{absdiff()}\spxextra{keras\_video.flow.OpticalFlowGenerator method}}

\begin{fulllineitems}
\phantomsection\label{\detokenize{index:keras_video.flow.OpticalFlowGenerator.absdiff}}\pysiglinewithargsret{\sphinxbfcode{\sphinxupquote{absdiff}}}{\emph{images}}{}
Get absolute differences between 2 images

\end{fulllineitems}

\index{diff\_mask() (keras\_video.flow.OpticalFlowGenerator method)@\spxentry{diff\_mask()}\spxextra{keras\_video.flow.OpticalFlowGenerator method}}

\begin{fulllineitems}
\phantomsection\label{\detokenize{index:keras_video.flow.OpticalFlowGenerator.diff_mask}}\pysiglinewithargsret{\sphinxbfcode{\sphinxupquote{diff\_mask}}}{\emph{images}}{}
Get absolute diff mask, then merge frames and apply the mask

\end{fulllineitems}

\index{flow\_mask() (keras\_video.flow.OpticalFlowGenerator method)@\spxentry{flow\_mask()}\spxextra{keras\_video.flow.OpticalFlowGenerator method}}

\begin{fulllineitems}
\phantomsection\label{\detokenize{index:keras_video.flow.OpticalFlowGenerator.flow_mask}}\pysiglinewithargsret{\sphinxbfcode{\sphinxupquote{flow\_mask}}}{\emph{images}}{}
Get optical flow on images, then merge images and apply the mask

\end{fulllineitems}

\index{get\_validation\_generator() (keras\_video.flow.OpticalFlowGenerator method)@\spxentry{get\_validation\_generator()}\spxextra{keras\_video.flow.OpticalFlowGenerator method}}

\begin{fulllineitems}
\phantomsection\label{\detokenize{index:keras_video.flow.OpticalFlowGenerator.get_validation_generator}}\pysiglinewithargsret{\sphinxbfcode{\sphinxupquote{get\_validation\_generator}}}{}{}
Return the validation generator if you’ve provided split factor

\end{fulllineitems}

\index{make\_optical\_flow() (keras\_video.flow.OpticalFlowGenerator method)@\spxentry{make\_optical\_flow()}\spxextra{keras\_video.flow.OpticalFlowGenerator method}}

\begin{fulllineitems}
\phantomsection\label{\detokenize{index:keras_video.flow.OpticalFlowGenerator.make_optical_flow}}\pysiglinewithargsret{\sphinxbfcode{\sphinxupquote{make\_optical\_flow}}}{\emph{images}}{}
Process Farneback Optical Flow on images

\end{fulllineitems}


\end{fulllineitems}



\chapter{Indices and tables}
\label{\detokenize{index:indices-and-tables}}\begin{itemize}
\item {} 
\DUrole{xref,std,std-ref}{genindex}

\item {} 
\DUrole{xref,std,std-ref}{modindex}

\item {} 
\DUrole{xref,std,std-ref}{search}

\end{itemize}


\renewcommand{\indexname}{Python Module Index}
\begin{sphinxtheindex}
\let\bigletter\sphinxstyleindexlettergroup
\bigletter{k}
\item\relax\sphinxstyleindexentry{keras\_video}\sphinxstyleindexpageref{index:\detokenize{module-keras_video}}
\item\relax\sphinxstyleindexentry{keras\_video.flow}\sphinxstyleindexpageref{index:\detokenize{module-keras_video.flow}}
\item\relax\sphinxstyleindexentry{keras\_video.generator}\sphinxstyleindexpageref{index:\detokenize{module-keras_video.generator}}
\item\relax\sphinxstyleindexentry{keras\_video.sliding}\sphinxstyleindexpageref{index:\detokenize{module-keras_video.sliding}}
\end{sphinxtheindex}

\renewcommand{\indexname}{Index}
\printindex
\end{document}